\documentclass{acm_proc_article-sp}

\begin{document}

\title{Extended Related Works of "Fast Counting of Triangles in Large Real Networks without counting:}
\subtitle{Algorithms and Laws"}

\numberofauthors{1}
\author{
\alignauthor
Amanda Wulf\\
       \affaddr{University of Texas at San Antonio}\\
       \affaddr{1 UTSA Circle}\\
       \affaddr{San Antonio, Texas 78240}\\
       \email{amandakwulf@gmail.com}
}

\maketitle
\begin{abstract}
This paper is an extended related works and background explanation for
\cite{original}.
\end{abstract}

\section{Introduction}
This paper's goal is to provide an extended background and related works for
\cite{original}. \cite{original} is a paper detailing a method of triangle
estimation in real-world graphs using a mathematical relationship between the
number of eigenvalues of real-world graphs and their triangle count.

\section{Graph Theory Background}
A graph G is defined as a way of encoding pairwise relationships among a set of
objects. It consists of a collection of nodes V and a collection of edges E, each
of which joins two nodes. Each edge v is and ordered pair (u, v). An edge e is
said to leave node u and enter node v. Nodes are also called vertices
\cite{kleinberg}.

Graphs can be either directed (meaning an edge can only be traversed in a
certain direction) or undirected \cite{kleinberg}. In this paper, the graphs we
consider are undirected.

A self-edge (or loop) is a node that has an edge connecting it to itself
\cite{diestel}.

Two vertices are adjacent if they have an edge connecting them. A fully
connected graph (also known as a complete graph) is a graph in which all
vertices are pairwise adjacent \cite{diestel}.

A triangle is defined to be a set of three fully connected nodes in an
undirected graph without self-edges \cite{original}.

\section{Triangle Counting and Listing Works}

\subsection{Count all triangles}
Schank and Wagner present an overview of triangle algorithms. They divide
triangle algorithms into either counting or listing algorithms, where counting
algorithms simply produce the number of triangles, while listing algorithms
produce a list of every triangle present in the graph \cite{schank}.

The brute force method they present, which traverses all nodes and counts every
set of three connected nodes, is called \textit{node-iterator}.

\subsection{Streaming and semi-streaming algorithms}
A streaming algorithm is an algorithm that handles data that arrives in a
data stream. Bar-Yossef et. al. have developed a streaming algorithm to
approximate the number of triangles in a large graph. This algorithm has the
advantage of being space efficient and only requiring a single pass over the
data \cite{baryossef}. 

\subsection{Optional: Other papers we were handed out in class}
Will fill this in last if I need more pages

\section{Background Works}

\subsection{Eigenvalues}
For a matrix A, given the expression $Ax = \lambda x$, where $x$ is a vector and
$\lambda$ is a real or complex number, $x$ is said to be the eigenvector of $A$
and $\lambda$ is said to be its eigenvalue. \cite{lovasz}

Here, add something about translating graph to matrix using graph theory book

Lanczos
Mihail and Papadimitriou
Harary and Schwenk

\section{Method}
The main idea of \cite{original} is the fact that the total number of triangles
in a graph is proportional to the sum of cubes of its adjacency matrix
eigenvalues.

\section{Conclusion}
This has been an explanation and related works for \cite{original}. This paper
has explained the background and related works so that \cite{original} is
easier to understand.

%
% The following two commands are all you need in the
% initial runs of your .tex file to
% produce the bibliography for the citations in your paper.
\bibliographystyle{abbrv}
\bibliography{sigproc}  % sigproc.bib is the name of the Bibliography in this case
% You must have a proper ".bib" file
%  and remember to run:
% latex bibtex latex latex
% to resolve all references
%
% ACM needs 'a single self-contained file'!
%
\end{document}
