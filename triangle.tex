\documentclass{acm_proc_article-sp}

\begin{document}

\title{Fast Counting of Triangles in Large Real Networks without counting:}
\subtitle{Algorithms and Laws}

\numberofauthors{1}
\author{
\alignauthor
Amanda Wulf\\
       \affaddr{University of Texas at San Antonio}\\
       \affaddr{1 UTSA Circle}\\
       \affaddr{San Antonio, Texas 78240}\\
       \email{amandakwulf@gmail.com}
}

\maketitle
\begin{abstract}
Abstract goes here
\end{abstract}

% TODO: Figure out categories
% A category with the (minimum) three required fields
\category{H.4}{Information Systems Applications}{Miscellaneous}
%A category including the fourth, optional field follows...
\category{D.2.8}{Software Engineering}{Metrics}[complexity measures, performance measures]

\section{Introduction}
Intro goes here. Explain that this is an extended related works for the original paper.

\section{Related Works}
This will be the meat of the paper.

\section{Method}
A pared-down version of the main ideas in the original paper.

\section{Conclusion}
Conclusion goes here.

%
% The following two commands are all you need in the
% initial runs of your .tex file to
% produce the bibliography for the citations in your paper.
\bibliographystyle{abbrv}
\bibliography{sigproc}  % sigproc.bib is the name of the Bibliography in this case
% You must have a proper ".bib" file
%  and remember to run:
% latex bibtex latex latex
% to resolve all references
%
% ACM needs 'a single self-contained file'!
%
\end{document}
