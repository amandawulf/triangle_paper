\documentclass{acm_proc_article-sp}

\begin{document}

\title{Extended Related Works of "Fast Counting of Triangles in Large Real Networks without counting:}
\subtitle{Algorithms and Laws"}

\numberofauthors{1}
\author{
\alignauthor
Amanda Wulf\\
       \affaddr{University of Texas at San Antonio}\\
       \affaddr{1 UTSA Circle}\\
       \affaddr{San Antonio, Texas 78240}\\
       \email{amandakwulf@gmail.com}
}

\maketitle
\begin{abstract}
This paper is an extended related works and background explanation for
\cite{original}.
\end{abstract}

\section{Introduction}
This paper's goal is to provide an extended background and related works for
\cite{original}. \cite{original} is a paper detailing a method of triangle
estimation in real-world graphs using a mathematical relationship between the
number of eigenvalues of real-world graphs and their triangle count.

\section{Motivation}
Include some works about why triangle counting is important
(clustering coefficient and transitivity ratio) here (probably from diestel).

\section{Graph Theory Background}
A graph G is defined as a way of encoding pairwise relationships among a set of
objects. It consists of a collection of nodes V and a collection of edges E, each
of which joins two nodes. Each edge v is and ordered pair (u, v). An edge e is
said to leave node u and enter node v. Nodes are also called vertices
\cite{kleinberg}.

Graphs can be either directed (meaning an edge can only be traversed in a
certain direction) or undirected \cite{kleinberg}. In this paper, the graphs we
consider are undirected.

A self-edge (or loop) is a node that has an edge connecting it to itself
\cite{diestel}.

Two nodes are adjacent if they have an edge connecting them. A fully
connected graph (also known as a complete graph) is a graph in which all
nodes are pairwise adjacent \cite{diestel}.

A triangle is defined to be a set of three fully connected nodes in an
undirected graph without self-edges \cite{original}.

The degree of a node is defined to be the number of edges that connect to it
\cite{diestel}.

A closed walk of a graph is a finite, loop-free, non-empty alternating sequence
of nodes and edges such that each edge is connected to each subsequent node,
and the end node is the same as the start node \cite{diestel}.

\section{Triangle Counting and Listing Works}

\subsection{Count all triangles}
Schank and Wagner present an overview of triangle algorithms. They divide
triangle algorithms into either counting or listing algorithms, where counting
algorithms simply produce the number of triangles, while listing algorithms
produce a list of every triangle present in the graph \cite{schank}.

The brute force method they present, which traverses all nodes and counts every
set of three connected nodes, is called \textit{node-iterator}.

\subsection{Streaming algorithms}
A streaming algorithm is an algorithm that handles data that arrives in a
data stream. Bar-Yossef et. al. have developed a streaming algorithm to
approximate the number of triangles in a large graph. This algorithm has the
advantage of being space efficient and only requiring a single pass over the
data \cite{baryossef}. 

\subsection{Semi-streaming algorithms}
Becchetti, Boldi, and Castillo present an algorithm to estimate the local
number of triangles in a graph (for each node, compute the number of triangles
that the node participates in). They chose to use a semi-streaming algorithm
for this because a streaming algorithm would constrain the memory usage too
much to be useful. The semi-streaming algorithm limits the passes over the data
to be at most $O(\log N)$ \cite{becchetti}.

\subsection{Algorithms that minimize disk I/O}
Triangle listing algorithms suffer from huge datasets and the fact that
in-memory algorithms are unable to handle looking at each node without
significant slowdowns from disk I/O. Chu and Cheng have solved this problem
using an algorithm that is designed to work on neighboring vertices as a whole
in order to minimize disk I/O accesses \cite{chu}.

\subsection{Algorithms that use MapReduce and Hadoop}
Exact triangle counting also has the issue of huge graphs many disk I/O
accesses. Suri and Vassilvitskii have developed an algorithm to calculate the
number of triangles using MapReduce on a computing cluster in order to mitigate
this issue \cite{suri}.

\section{Background Works}

\subsection{Adjacency matrices}
An adjacency matrix of a graph which has n nodes is a matrix $A = 
(a_{ij})_{nxn}$ is defined by $a_{ij} = 1$ if $v_iv_j \in E$ and $0$ otherwise
\cite{diestel}.  This gives us a way to do linear algebra operations on graphs.

\subsection{Eigenvalues}
For a matrix A, given the expression $Ax = \lambda x$, where $x$ is a vector and
$\lambda$ is a real or complex number, $x$ is said to be the eigenvector of $A$
and $\lambda$ is said to be its eigenvalue. \cite{lovasz}

\subsection{Eigenvalue calculations}
The Lanczos method is an iterative method of estimating the eigenvalues in a
matrix. It provides a good, fast estimate for large graphs with sparse
eigenvalues \cite{golub}.

\subsection{Power laws}
A power law is a J-shaped, highly skewed distribution function of some
empirical data with a long tail \cite{simon}.

A degree power law is a graph with a power law for graph degrees. In other
words, it has a few nodes with a high degree and many nodes with a low degree
\cite{mihail}.

If a graph has a degree power law, it also has an eigenvalue power law
\cite{mihail}. Degree power laws are common in real networks \cite{original},
and thus by extension eigenvalue power laws are also common. The fact that
these graphs have eigenvalue power laws makes the eigenvalues fast to calculate
because there are only a few important eigenvalues, and many unimportant
(small) eigenvalues, so the Lanczos method can be used to find the most
important eigenvalues and thus find a close approximation quickly \cite{original}.

\subsection{Number of closed walks in a graph}
In \cite{harary}, Harary and Schwenk prove that the number of closed walks of
length n in a graph is the sum of the nth powers of the graph's eigenvalues. By
extension, the number of closed walks of length 3 is the sum of cubes of the
graph's eigenvalues, and a closed walk of length 3 is a triangle.

\section{Method}
The main idea of \cite{original} is the fact that the total number of triangles
in a graph is proportional to the sum of cubes of its adjacency matrix
eigenvalues.

The proposed algorithm to calculate the number of triangles involves
iteratively using Lanczos method to calculate the eigenvalues, cubing and
summing them as we calculate them.

\section{Conclusion}
This has been an explanation and related works for \cite{original}. This paper
has explained the background and related works so that \cite{original} is
easier to understand.

%
% The following two commands are all you need in the
% initial runs of your .tex file to
% produce the bibliography for the citations in your paper.
\bibliographystyle{abbrv}
\bibliography{sigproc}  % sigproc.bib is the name of the Bibliography in this case
% You must have a proper ".bib" file
%  and remember to run:
% latex bibtex latex latex
% to resolve all references
%
% ACM needs 'a single self-contained file'!
%
\end{document}
