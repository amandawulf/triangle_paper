\documentclass{acm_proc_article-sp}

\begin{document}

\title{Extended Related Works of "Fast Counting of Triangles in Large Real Networks without counting:}
\subtitle{Algorithms and Laws"}

\numberofauthors{1}
\author{
\alignauthor
Amanda Wulf\\
       \affaddr{University of Texas at San Antonio}\\
       \affaddr{1 UTSA Circle}\\
       \affaddr{San Antonio, Texas 78240}\\
       \email{amandakwulf@gmail.com}
}

\maketitle
\begin{abstract}
Abstract goes here. Explain that this is an extended related works for the original paper.
\end{abstract}

\section{Introduction}
Intro goes here. Explain that this is an extended related works for the original paper.

\section{Related Triangle Works}

\subsection{Brute Force}
Check every permutation of three nodes to see if they are connected.

\subsection{Count all triangles}
Shank and Wagner

\subsection{Streaming and semi-streaming algorithms}
Bar-Yossef, Kumar, and Sivakumar

\subsection{Optional: Other papers we were handed out in class}
Will fill this in last if I need more pages

\section{Related Works for the Background of this Paper}

\subsection{Eigenvalues}
Lovasz
Lanczos
Mihail and Papadimitriou
Harary and Schwenk

\section{Method}
The main idea of \cite{original} is the fact that the total number of triangles
in a graph is proportional to the sum of cubes of its adjacency matrix
eigenvalues.

\section{Conclusion}
Conclusion goes here.

%
% The following two commands are all you need in the
% initial runs of your .tex file to
% produce the bibliography for the citations in your paper.
\bibliographystyle{abbrv}
\bibliography{sigproc}  % sigproc.bib is the name of the Bibliography in this case
% You must have a proper ".bib" file
%  and remember to run:
% latex bibtex latex latex
% to resolve all references
%
% ACM needs 'a single self-contained file'!
%
\end{document}
